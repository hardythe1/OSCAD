%\documentclass[12pt]{report}
%\usepackage{amsmath,graphicx,makeidx}

%\makeindex
%\begin{document}


\chapter{Introduction}

Electronic systems are an integral part of human life. They have simplified our lives to a great extent. Starting from small systems made of a few discrete components to the present day integrated circuits (ICs) with millions of logic gates, electronic systems have undergone a sea change. As a result, design of electronic systems too have become extremely difficult and time consuming. Thanks to a host of computer aided design tools, we have been able to come up with quick and efficient designs. These tools are called {\tt Electronic Design Automation} or {\tt EDA} \index{EDA! tools}tools.
\section {EDA design flow}
Let us see the steps involved in EDA.\index{EDA!design flow} In the first stage, the specifications of the system are laid out. These specifications are then converted to a design. The design could in the form of a circuit schematic, logical description using an HDL language etc. The design is then simulated and re-designed, if needed, to achieve the desired results. Once simulation achieves the specifications, the design is either converted to a PCB, a chip layout, or ported to an FPGA etc. The final product is again tested for specifications. The whole cycle is repeated until desired results are obtained.
\section{EDA Tools}
If you are building an electronic system, you would first design your circuit, make its schematic diagram, simulate it and finally convert it into a Printed Circuit Board (PCB) \index{PCB}. There are various tools available that would help you do this. Some of the popular EDA tools are those of {\tt Cadence}, {\tt Synopys}, {\tt Mentor Graphics}, {\tt Xilinx} etc. These are proprietary tools. There are some open source EDA tools like {\tt gEDA}, {\tt KiCad}, {\tt Ngspice} etc. 
\section{What is Oscad?}
Oscad is a free and open source EDA tool. \index{Oscad}It is an acronym for \textbf{O}pen \textbf{s}ource \textbf{c}omputer \textbf{a}ided \textbf{d}esign. Oscad is created using several open source software packages namely KiCad, Ngspice, Scilab and Python. Using Oscad, one can create circuit schematics, perform simulation and design PCB layouts. It can create or edit new device models, and create or edit subcircuits for simulation. It also has a Scilab based Mini Circuit Simulator (SMCSim) which is
capable of giving the circuit equations for each simulation step. This feature is unique to Oscad. 
\section{Why Oscad?}
Proprietary EDA tools are fairly comprehensive and high end. But their licences are very expensive. The main drawback of the open source tools is that they are not comprehensive. Some of them are capable of PCB design (e.g., {\tt KiCad}) while some of them are capable of performing simulations (e.g., {\tt gEDA}). There is no well known open source software that can perform circuit design, layout design and circuit simulation together. Oscad is capable of doing all of the above. This is why Oscad is very important to students, teachers and other professionals who would want to study and/or design electronic systems. Oscad is also very useful for entrepreneurs and small scale enterprises who do not have the capability to invest on heavily priced proprietary tools.
\section{Structure of the book}
This book introduces Oscad to the reader and illustrates all the features of Oscad with examples. Chapter \ref{install} gives step by step instructions to install Oscad on your computer. The software architecture of Oscad is presented in Chapter \ref{struct}. 
 Chapter \ref{get} gets you started with Oscad. It takes you through a tour of Oscad with the help of a simple RC circuit example. Chapter \ref{schem} explains how to create circuit schematics using Oscad, in detail using examples. Chapter \ref{sim} illustrates how to simulate circuits using Oscad. Chapter \ref{pcb} explains PCB design using Oscad, in detail. The advanced feature of Oscad like Model builder, subcircuit builder and Scilab based simulations are covered in the Apter \ref{adv}. Chapter \ref{spo} describes the spoken tutorials on Oscad and contains instructions to use them. Appendix A presents examples from the book {\tt Microelectronic Circuits} by Sedra and Smith that have been worked out using Oscad.









%\printindex
%\end{document}
